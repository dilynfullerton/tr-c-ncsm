\documentclass{article}
\usepackage{amsmath}
\usepackage{amsfonts}
\usepackage{hyperref}
\usepackage[margin=1.0in]{geometry}
\usepackage{graphicx}
\usepackage{float}
\setlength\parindent{0pt}

\begin{document}
\title{$T_{\rm{rel}}$ and the Importance-Truncation Code}
\author{Dilyn Fullerton}
\date{2016-06-01}
\maketitle

\section{Introduction}
The purpose of this composition is to record the exact modifications
I made to the No-Core Shell Model importance truncation code
(\texttt{it-code-111815.f}) and the motivation behind them.

\section{Two definitions of $T_{\rm{rel}}$}

For an $A$-particle system with an effective mass $A_{\rm{eff}}$,
relative kinetic energy, $T_{\rm{rel}}$, may be defined in two different
ways.

\subsection{Definition 1}

The first way to define relative kinetic energy is by subtracting
the center of mass kinetic energy from the total kinetic energy.
The sums are over the number of particles, $A$, but an effective
particle number, $A_{\rm{eff}}$, is used in the center of mass term.

\begin{equation*}
  \begin{aligned}
    T_{\rm{rel},1} &= T - T_{\rm{CM}} \\
    &= \displaystyle\sum_{i=1}^A \cfrac{\textbf{p}_i^2}{2m}
    ~- \cfrac{\textbf{p}_{\rm{tot}}^2}{2mA_{\rm{eff}}}, \\
  \end{aligned}
\end{equation*}

where

\begin{equation*}
  \textbf{p}_{\rm{tot}} = \displaystyle\sum_{i=1}^A \textbf{p}_i.
\end{equation*}

This can be written in the form

\begin{equation}\label{eq:def1}
  T_{\rm{rel},1} =
  \left(1-\frac{1}{A_{\rm{eff}}}\right) \displaystyle\sum_{i=1}^A\tau_i
  - \frac{1}{2A_{\rm{eff}}}
  \displaystyle\sum_{i\neq j=1}^A\tau_{ij},
\end{equation}

where

\begin{equation}\label{eq:ti,tij}
  \tau_i = \cfrac{\textbf{p}_i^2}{2m}, \quad
  \tau_{ij} = \cfrac{\textbf{p}_i\cdot\textbf{p}_j}{m}.
\end{equation}

\subsection{Definition 2}

The second way to define relative kinetic energy is by adding the
relative momentum differences of all of the particles. The sum, again,
is over the true particle number, $A$, while $A_{\rm{eff}}$ is used as
the effective particle number for defining the mass.

\begin{equation*}
  T_{\rm{rel},2}= \displaystyle\sum_{i<j=1}^A
  \cfrac{(\textbf{p}_i-\textbf{p}_j)^2}{2mA_{\rm{eff}}}
\end{equation*}

The can be written in the form

\begin{equation}\label{eq:def2}
  T_{\rm{rel},2}= \cfrac{A-1}{A_{\rm{eff}}} \displaystyle\sum_{i=1}^A\tau_i
  - \cfrac{1}{2A_{\rm{eff}}} \displaystyle\sum_{i\neq j=1}^A\tau_{ij},
\end{equation}

where $\tau_i$ and $\tau_{ij}$ are defined as in (\ref{eq:ti,tij}).

The two definitions are clearly equal when $A_{\rm{eff}}=A$; however,
they are different in general.

\section{Operators based on Definition 1}

We find the operators that stem from defining $T_{\rm{rel}}$ according
to (\ref{eq:def1}).

\subsection{Mixed 1,2-body operator}

Equation (\ref{eq:def1}) translates easily into a 1,2-body operator.

\begin{equation}\label{eq:def1op1,2}
  \hat{T}_{\rm{rel},1}^{(1,2)} = \left(1-\frac{1}{A_{\rm{eff}}}\right)
  \displaystyle\sum_{abJT}\tau_{ab}^{JT}a_a^{\dagger}a_b
  -\frac{1}{2A_{\rm{eff}}} \displaystyle\sum_{\substack{abcd\\JT}}
  \tau_{abcd}^{JT}a_a^{\dagger}a_b^{\dagger}a_ca_d,
\end{equation}

where

\begin{equation}\label{eq:tab,tabcd}
  \tau_{ab}^{JT} = \left\langle a\middle|
  \frac{\textbf{p}^2}{2m} \middle|b\right\rangle_{JT}, \quad
  \tau_{abcd}^{JT} = \left\langle ab\middle|
  \frac{\textbf{p}_1\cdot\textbf{p}_2}{m} \middle|cd\right\rangle_{JT}
\end{equation}
\begin{equation*}
  \left\langle a\middle|
  \frac{\textbf{p}^2}{2m} \middle|b\right\rangle_{JT} = 
  \begin{cases}
    \frac{1}{2}(2n_a+l_a+3/2),
    & n_a=n_b, l_a=l_b, j_a=j_b\\
    \frac{1}{2}\sqrt{n_a(n_a+l_a+1/2)}, 
    & n_a=n_b+1, l_a=l_b, j_a=j_b\\
    0, & \rm{else}
  \end{cases}.
\end{equation*}

\subsection{Fully 2-body operator}

The 1,2-body operator defined above (\ref{eq:def1op1,2}) can be
refactored into a fully 2-body operator.

\begin{equation}\label{eq:def1op2}
  \begin{aligned}
    \hat{T}_{\rm{rel},1}^{(2)} &= \left(1-\frac{1}{A_{\rm{eff}}}\right)
    \displaystyle\sum_{\substack{abcd\\JT}} \frac{\alpha}{4(A-1)}
    \tilde{\tau}_{abcd}^{JT} a_a^{\dagger}a_b^{\dagger}a_ca_d
    -\frac{1}{2A_{\rm{eff}}} \displaystyle\sum_{\substack{abcd\\JT}}
    \tau_{abcd}^{JT} a_a^{\dagger}a_b^{\dagger}a_ca_d\\
    &= \frac{2}{A_{\rm{eff}}} \displaystyle\sum_{\substack{abcd\\JT}}
    \left(
    \frac{\alpha}{8}\cdot\cfrac{A_{\rm{eff}}-1}{A-1}
    \tilde{\tau}_{abcd}^{JT} -\frac{1}{4}\tau_{abcd}^{JT}
    \right) a_a^{\dagger}a_b^{\dagger}a_ca_d,
  \end{aligned}
\end{equation}

where

\begin{equation}\label{eq:ttil}
  \tilde{\tau}_{abcd}^{JT} = N_{abcd}
  \left( \rho_{abcd}^{JT} -(-1)^{\phi_{ab}} \rho_{abdc}^{JT} \right)
  \bigtriangleup(j_a,j_b,J) \bigtriangleup(t_a,t_b,T)
\end{equation}
\begin{gather*}
  \alpha = 2\\
  N_{abcd} = \sqrt[-1]{(1-\delta_{ab})(1-\delta_{cd})}\\
  \rho_{abcd}^{JT} =
  \tau_{ac}^{JT}\delta_{bd} + \tau_{bd}^{JT}\delta_{ac}\\
  \phi_{ab} = j_a+j_b-J + t_a+t_b-T\\
  \bigtriangleup(a,b,c) = 
  \begin{cases}
    1, & 0\leq c\leq a+b\\
    0, & \rm{else}\\
  \end{cases}
\end{gather*}

\section{Operators based on Definition 2}

We find the operators that stem from defining $T_{\rm{rel}}$ according
to (\ref{eq:def2}).

\subsection{Mixed 1,2-body operator}

Equation (\ref{eq:def2}) translates easily into a 1,2-body operator.

\begin{equation}\label{eq:def2op1,2}
  \hat{T}_{\rm{rel},2}^{(1,2)} = \cfrac{A-1}{A_{\rm{eff}}}
  \displaystyle\sum_{abJT}\tau_{ab}^{JT}a_a^{\dagger}a_b
  -\frac{1}{2A_{\rm{eff}}} \displaystyle\sum_{\substack{abcd\\JT}}
  \tau_{abcd}^{JT}a_a^{\dagger}a_b^{\dagger}a_ca_d.
\end{equation}

\subsection{Fully 2-body operator}

The 1,2-body operator defined above (\ref{eq:def2op1,2}) can be
refactored into a fully 2-body operator.

\begin{equation}\label{eq:def2op2}
  \begin{aligned}
    \hat{T}_{\rm{rel},2}^{(2)} &= \left(\frac{A-1}{A_{\rm{eff}}}\right)
    \displaystyle\sum_{\substack{abcd\\JT}} \frac{\alpha}{4(A-1)}
    \tilde{\tau}_{abcd}^{JT} a_a^{\dagger}a_b^{\dagger}a_ca_d
    -\frac{1}{2A_{\rm{eff}}} \displaystyle\sum_{\substack{abcd\\JT}}
    \tau_{abcd}^{JT} a_a^{\dagger}a_b^{\dagger}a_ca_d\\
    &= \frac{2}{A_{\rm{eff}}} \displaystyle\sum_{\substack{abcd\\JT}}
    \left( \frac{\alpha}{8}\tilde{\tau}_{abcd}^{JT}
    -\frac{1}{4}\tau_{abcd}^{JT} \right) a_a^{\dagger}a_b^{\dagger}a_ca_d.
  \end{aligned}
\end{equation}

\section{Importance truncation code}

The kinetic energy term in the \texttt{it-code-111815.f} file is based
on Defintion 2. This is also multiplied by $2/A_{\rm{eff}}$ in the
code, so we define

\begin{equation}
  (\tau_{\rm{file}})_{abcd}^{JT} = \frac{1}{4}
  \left( \frac{\alpha}{2} \tilde{\tau}_{abcd}^{JT} - \tau_{abcd}^{JT}
  \right),
\end{equation}

so that (\ref{eq:def2op2}) can be simplified to

\begin{equation}
  \hat{T}_{\rm{rel},2}^{(2)} = \frac{2}{A_{\rm{eff}}}
  \displaystyle\sum_{\substack{abcd\\JT}} (\tau_{\rm{file}})_{abcd}^{JT}
  a_a^{\dagger}a_b^{\dagger}a_ca_d.
\end{equation}

We also define

\begin{equation}
  (\tau_{\rm{add}})_{abcd}^{JT} = \frac{\alpha}{8}\cdot
  \cfrac{A_{\rm{eff}}-A}{A-1}
  \tilde{\tau}_{abcd}^{JT},
\end{equation}

so that (\ref{eq:def1op2}), the two-body operator based on
Definition 1, can be simplified to

\begin{equation}
  \hat{T}_{\rm{rel},1}^{(2)} = \frac{2}{A_{\rm{eff}}}
  \displaystyle\sum_{\substack{abcd\\JT}} \left[ 
  (\tau_{\rm{file}})_{abcd}^{JT} + (\tau_{\rm{add}})_{abcd}^{JT}
  \right] a_a^{\dagger}a_b^{\dagger}a_ca_d.
\end{equation}

Now since Definition 1 for relative kinetic energy is the definition
that is truely fundamental, $(\tau_{\rm{add}})_{abcd}^{JT}$ defines
the modification to the code in order to produce the correct
kinetic energy when $A_{\rm{eff}}\neq A$.
This is the modification I have implemented in
\texttt{it-code-111815.f}.

\section{References}

\begin{description}
\item[Original and modified NCSM codes:]\hfill\\
  \url{https://github.com/dilynfullerton/tr-c-ncsm/}
\item[More general discussion of definitions of $T_{\rm{rel}}$:]\hfill\\
  \url{http://wiki.triumf.ca/wiki/IMSRG/index.php/Relative_kinetic_energy}
\item[Promoting a one-body to a two-body operator:]\hfill\\
  \url{http://wiki.triumf.ca/wiki/IMSRG/index.php/Promoting_a_one-body_operator}
\item[Occupation number representation of 1- and 2-body operators:]\hfill\\
  Suhonen. \textit{From Nucleons to Nucleus}. Chapter 4.2.
  
\end{description}

\end{document}
