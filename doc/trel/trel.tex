\documentclass{article}
\usepackage{amsmath}
\usepackage{amsthm}
\usepackage{amsfonts}
\usepackage{hyperref}
\usepackage[margin=1.0in]{geometry}
\usepackage{graphicx}
\usepackage{float}
\setlength\parindent{0pt}


%% Defintions
\def\aeff{A_{\rm{eff}}}
\def\trel{T_{\rm{rel}}}
\def\treli_#1{T_{\rm{rel},#1}}
\def\trelop{\hat{T}_{\rm{rel}}}
\def\trelopi_#1^#2{\hat{T}_{\rm{rel}, #1}^{#2}}
\def\momvect{\boldsymbol{p}}
\def\posvect{\boldsymbol{x}}
\def\creop{a^{\dagger}}
\def\annop{a}
\def\matelt#1#2#3{\left\langle{#1}\middle|{#2}\middle|{#3}\right\rangle}

\newtheorem{definition}{Definition}[section]

%% Document start
\begin{document}
\title{$T_{\rm{rel}}$ and the NCSD Importance-Truncation Code}
\author{Dilyn Fullerton}
\date{2016--09--30}
\maketitle

\tableofcontents

\section{Introduction}
The purpose of this composition is to record the exact modifications
I made to the No-Core Shell Model importance truncation code
(\texttt{it-code-111815.f}) and the motivation behind them.

\section{Two definitions of $\trel$}
For an $A$-particle system with an effective mass $\aeff$,
relative kinetic energy, $\trel$, may be defined in two different
ways~\cite{treldefs}.
\\\\
%% Definition 1
The first way to define relative kinetic energy is by subtracting
the center of mass kinetic energy from the total kinetic energy.
The sums are over the number of particles, $A$, but an effective
particle number, $\aeff$, is used in the center of mass term.

\begin{definition}\label{def:trel1}
  Let the relative kinetic energy with respect to an effective mass $\aeff$
  be defined by the total kinetic energy minus the kinetic energy of
  the center of mass, where the effective term is used in the center of
  mass kinetic energy.
  \begin{equation*}
    \treli_1 = T^A - T_{\rm{CM}}^{\aeff}
  \end{equation*}
\end{definition}

Expanding this in terms of particle momenta gives
\begin{equation*}
  \treli_1 =
  \left(\displaystyle\sum_{i=1}^A \cfrac{\momvect_i^2}{2m}\right)
  ~- \cfrac{\momvect_{\rm{tot}}^2}{2m\aeff},
\end{equation*}

where

\begin{equation*}
  \momvect_{\rm{tot}} = \displaystyle\sum_{i=1}^A \momvect_i.
\end{equation*}

This can be rewritten into the form

\begin{equation}\label{eq:trel1}
  \treli_1 =
  \left(1-\frac{1}{\aeff}\right) \displaystyle\sum_{i=1}^A\tau_i
  - \frac{1}{2\aeff}
  \displaystyle\sum_{i\neq j=1}^A\tau_{ij},
\end{equation}

where

\begin{equation}\label{eq:ti,tij}
  \tau_i = \cfrac{\momvect_i^2}{2m}, \quad
  \tau_{ij} = \cfrac{\momvect_i\cdot\momvect_j}{m}.
\end{equation}

%% Definition 2
The second way to define relative kinetic energy is by adding the
relative momentum differences of all of the particles. The sum, again,
is over the true particle number, $A$, while $\aeff$ is used as
the effective particle number for defining the mass.

\begin{definition}\label{def:trel2}
  Let the relative kinetic energy with respect to an effective mass $\aeff$
  be defined by the sum of differences of momenta of individual particles
  according to
  \begin{equation*}
    \treli_2 = \displaystyle\sum_{i<j=1}^A
    \cfrac{{(\momvect_i-\momvect_j)}^2}{2m\aeff}
  \end{equation*}
\end{definition}

This can be rewritten into the form

\begin{equation}\label{eq:trel2}
  \treli_2 = \cfrac{A-1}{\aeff} \displaystyle\sum_{i=1}^A\tau_i
  ~-\cfrac{1}{2\aeff} \displaystyle\sum_{i\neq j=1}^A\tau_{ij},
\end{equation}

where $\tau_i$ and $\tau_{ij}$ are defined as in Equation (\ref{eq:ti,tij}).
\\\\
The two definitions are clearly equal when $\aeff=A$; however,
they are different in general.

%% Mixed 1,2 operators
\section{Mixed 1,2 operators based on $\trel$ definitions}
We can translate the coordinate forms of the the two kinetic energy
definitions into their occumpation number representation according to
the following equations~\cite{suhonen}.
\\\\
One body operator in occupation number representation:
\begin{equation}\label{eq:ocn1bd}
  \begin{aligned}
    &T = \displaystyle\sum_{i=1}^A t_i(\posvect_i) =
    \displaystyle\sum_{ab}t_{ab}\creop_a\annop_b,\\
    &t_{ab} = \matelt{a}{T}{b}
  \end{aligned}
\end{equation}

Two body operator in occupation number representation:
\begin{equation}\label{eq:ocn2bd}
  \begin{aligned}
    &V = \frac{1}{2}\displaystyle\sum_{i \neq j}
    v\left(\posvect_i, \posvect_j\right) =
    \frac{1}{2}\displaystyle\sum_{abcd}
    v_{abcd}\creop_a\creop_b\annop_d\annop_c,\\
    &v_{abcd} = \matelt{ab}{V}{cd}
  \end{aligned}
\end{equation}

%% Mixed 1,2 operators for Trel 1
\subsection{Mixed 1,2-body operator for Definition~\ref{def:trel1}}
The following is the mixed 1,2-body operator based on Definition
\ref{def:trel1}. The operator is obtained by translating Equation
(\ref{eq:trel1}) to occupation number representation using Equations
(\ref{eq:ocn1bd}) and (\ref{eq:ocn2bd}).

\begin{equation}\label{eq:trel1op1,2}
  \trelopi_1^{(1,2)} = \left(1-\frac{1}{\aeff}\right)
  \displaystyle\sum_{abJT}\tau_{ab}^{JT}\creop_a\annop_b
  -\frac{1}{2\aeff} \displaystyle\sum_{\substack{abcd\\JT}}
  \tau_{abcd}^{JT}\creop_a\creop_b\annop_d\annop_c,
\end{equation}

where

\begin{equation}\label{eq:tab,tabcd}
  \tau_{ab}^{JT} = \matelt{a}{\frac{\momvect^2}{2m}}{b}_{JT}, \quad
  \tau_{abcd}^{JT} = \matelt{ab}{\frac{\momvect_1\cdot\momvect_2}{m}}{cd}_{JT},
\end{equation}

and

\begin{equation*}
  \matelt{a}{\frac{\momvect^2}{2m}}{b}_{JT} = 
  \begin{cases}
    \frac{1}{2}(2n_a+l_a+3/2),
    & n_a=n_b, l_a=l_b, j_a=j_b\\
    \frac{1}{2}\sqrt{n_a(n_a+l_a+1/2)}, 
    & n_a=n_b+1, l_a=l_b, j_a=j_b\\
    0, & \rm{else}
  \end{cases}.
\end{equation*}

\subsection{Mixed 1,2-body operator for Definition~\ref{def:trel2}}
The following is the mixed 1,2-body operator based on Definition
\ref{def:trel2}. The operator is obtained by translating Equation
(\ref{eq:trel2}) to occupation number representation using Equations
(\ref{eq:ocn1bd}) and (\ref{eq:ocn2bd}).

\begin{equation}\label{eq:trel2op1,2}
  \trelopi_2^{(1,2)} = \cfrac{A-1}{\aeff}
  \displaystyle\sum_{abJT}\tau_{ab}^{JT}\creop_a\annop_b
  -\frac{1}{2\aeff} \displaystyle\sum_{\substack{abcd\\JT}}
  \tau_{abcd}^{JT}\creop_a\creop_b\annop_d\annop_c
\end{equation}

where $\tau_{ab}^{JT}$ and $\tau_{abcd}^{JT}$ are as defined in Equation
(\ref{eq:tab,tabcd}).


%% Fully 2 body operators
\section{Fully 2-body operators from mixed 1,2 body operators}
We now look to translate the 1,2-body $\trelop$ operators into fully 2-body
operators. A general 1,2-body operator $\mathcal{O}$ may be translated into
a fully 2-body operator according to the equation

\begin{equation*}
  \mathcal{O} = \displaystyle\sum_{ij}o_{ij}\creop_i\annop_j =
  \frac{1}{4}\displaystyle\sum_{ijkl}\tilde{o}_{ijkl}
  \creop_i\creop_j\annop_l\annop_k,
\end{equation*}

where $\tilde{o}_{ijkl}$ is the antisymmetrized two-body matrix element
defined by

\begin{equation*}
  \tilde{o}_{ijkl} = \mathcal{N}\left(
  o_{ik}\delta_{jl} + o_{jl}\delta_{ik} -
  \left( o_{il}\delta_{jk} + o_{jk}\delta_{il} \right)
  \right)
\end{equation*}

for some normalization $\mathcal{N}$~\cite{promoting1to2}.
In the case where this is coupled to some $J$ and $T$ (as here), the
transformation for an operator $\mathcal{T}$ is

\begin{equation}\label{eq:12to2}
  \mathcal{T} = \displaystyle\sum_{ij}\tau_{ij}^{JT}\creop_i\annop_j =
  \frac{1}{4}\cdot\frac{1}{A-1}\displaystyle\sum_{ijkl}\tilde{\tau}_{ijkl}^{JT}
  \creop_i\creop_j\annop_l\annop_k,
\end{equation}

where

\begin{equation}\label{eq:12to2op}
  \tilde{\tau}_{ijkl}^{JT} = \mathcal{N}_{abcd} \left(
  \rho_{abcd}^{JT} - {(-1)}^{\phi_{ab}^{JT}}\rho_{abdc}^{JT}
  \right)
  \bigtriangleup\left(j_a, j_b, J\right)
  \bigtriangleup\left(t_a, t_b, T\right),
\end{equation}

and I have defined

\begin{gather*}
  \mathcal{N}_{abcd} = \sqrt[-1]{(1-\delta_{ab})(1-\delta_{cd})}\\
  \rho_{abcd}^{JT} = o_{ac}^{JT}\delta_{bd} + o_{bd}^{JT}\delta_{ac}\\
  \phi_{ab}^{JT} = j_a+j_b-J + t_a+t_b-T\\
  \bigtriangleup(a,b,c) = 
  \begin{cases}
    1, & |a-b|\leq c\leq a+b\\
    0, & \rm{else}\\
  \end{cases}
\end{gather*}

%% Fully 2-body operator for trel 1
\subsection{Fully 2-body operator for Definition~\ref{def:trel1}}
The following is the fully 2-body operator based on Definition~\ref{def:trel1}.
The operator is obtained by translating the mixed 1,2-body operator in
Equation (\ref{eq:trel1op1,2}) using Equation (\ref{eq:12to2}).

\begin{equation}\label{eq:trel1op2}
  \begin{aligned}
    \trelopi_1^{(2)} &= \left(1-\frac{1}{\aeff}\right)
    \displaystyle\sum_{\substack{abcd\\JT}} \frac{1}{4(A-1)}
    \tilde{\tau}_{abcd}^{JT} \creop_a\creop_b\annop_d\annop_c
    -\frac{1}{2\aeff} \displaystyle\sum_{\substack{abcd\\JT}}
    \tau_{abcd}^{JT} \creop_a\creop_b\annop_d\annop_c\\
    &= \frac{2}{\aeff} \displaystyle\sum_{\substack{abcd\\JT}}
    \left(
    \frac{1}{8}\cdot\cfrac{\aeff-1}{A-1}
    \tilde{\tau}_{abcd}^{JT} -\frac{1}{4}\tau_{abcd}^{JT}
    \right) \creop_a\creop_b\annop_d\annop_c,
  \end{aligned}
\end{equation}

where $\tau_{abcd}^{JT}$ is as defined in Equation (\ref{eq:tab,tabcd}) and
$\tilde{\tau}_{abcd}^{JT}$ is as defined in Equation (\ref{eq:12to2op}).

%% Fully 2-body operator for trel 2
\subsection{Fully 2-body operator for Definition~\ref{def:trel2}}
The following is the fully 2-body operator based on Definition~\ref{def:trel2}.
The operator is obtained by translating the mixed 1,2-body operator in
Equation (\ref{eq:trel2op1,2}) using Equation (\ref{eq:12to2}).

\begin{equation}\label{eq:trel2op2}
  \begin{aligned}
    \trelopi_2^{(2)} &= \left(\frac{A-1}{\aeff}\right)
    \displaystyle\sum_{\substack{abcd\\JT}} \frac{1}{4(A-1)}
    \tilde{\tau}_{abcd}^{JT} \creop_a\creop_b\annop_d\annop_c
    -\frac{1}{2\aeff} \displaystyle\sum_{\substack{abcd\\JT}}
    \tau_{abcd}^{JT} \creop_a\creop_b\annop_d\annop_c\\
    &= \frac{2}{\aeff} \displaystyle\sum_{\substack{abcd\\JT}}
    \left( \frac{1}{8}\tilde{\tau}_{abcd}^{JT}
    -\frac{1}{4}\tau_{abcd}^{JT} \right) \creop_a\creop_b\annop_d\annop_c.
  \end{aligned}
\end{equation}

\section{Importance truncation code}

The kinetic energy term in the \texttt{it-code-111815.f} file is based
on Defintion~\ref{def:trel2}. This is also multiplied by $2/A_{\rm{eff}}$ in
the code, so we define

\begin{equation}
  {(\tau_{\rm{file}})}_{abcd}^{JT} = \frac{1}{4}
  \left( \frac{1}{2} \tilde{\tau}_{abcd}^{JT} - \tau_{abcd}^{JT} \right),
\end{equation}

%% so that Equation (\ref{eq:trel2op2}) can be simplified to

\begin{equation}
  \trelopi_2^{(2)} = \frac{2}{\aeff}
  \displaystyle\sum_{\substack{abcd\\JT}} {(\tau_{\rm{file}})}_{abcd}^{JT}
  \creop_a\creop_b\annop_d\annop_c.
\end{equation}

We also define

\begin{equation}
  {(\tau_{\rm{add}})}_{abcd}^{JT} = \frac{1}{8}\cdot
  \cfrac{\aeff-A}{A-1} \tilde{\tau}_{abcd}^{JT},
\end{equation}

so that Equation (\ref{eq:trel1op2}), the two-body operator based on
Definition~\ref{def:trel2}, can be simplified to

\begin{equation}
  \trelopi_1^{(2)} = \frac{2}{\aeff}
  \displaystyle\sum_{\substack{abcd\\JT}} \left[ 
  {(\tau_{\rm{file}})}_{abcd}^{JT} + {(\tau_{\rm{add}})}_{abcd}^{JT}
  \right] \creop_a\creop_b\annop_d\annop_c.
\end{equation}

Now since Definition~\ref{def:trel1} for relative kinetic energy is the
definition that we take to be truely fundamental,
${(\tau_{\rm{add}})}_{abcd}^{JT}$ defines
the modification to the code in order to produce the correct
kinetic energy when $\aeff\neq A$.
This is the modification I \textit{will soon} implement in
\texttt{it-code-111815.f}~\cite{github}.

\bibliographystyle{amsplain}
\bibliography{trel}

\end{document}
